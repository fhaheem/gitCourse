\documentclass[11pt,letterpaper]{article}

% ============================================
% PACKAGES
% ============================================
\usepackage[utf8]{inputenc}
\usepackage[margin=1in]{geometry}
\usepackage{listings}
\usepackage{xcolor}
\usepackage{fancyhdr}
\usepackage{tcolorbox}
\usepackage{graphicx}
\usepackage{titlesec}

\setlength{\parindent}{0pt}
\setlength{\parskip}{0.4em}

% ============================================
% CODE STYLING
% ============================================
\definecolor{codegray}{rgb}{0.5,0.5,0.5}
\definecolor{backcolour}{rgb}{0.95,0.95,0.92}
\definecolor{codeorange}{RGB}{191, 97, 0}
\definecolor{yamlkey}{RGB}{0, 0, 255}
\definecolor{yamlcomment}{RGB}{0, 128, 0}

\lstdefinestyle{codestyle}{
    backgroundcolor=\color{backcolour},
    basicstyle=\ttfamily\footnotesize,
    commentstyle=\color{yamlcomment},
    keywordstyle=\color{yamlkey},
    breaklines=true,
    keepspaces=true,
    showspaces=false,
    showstringspaces=false,
    showtabs=false,
    numbers=left,
    numberstyle=\tiny\color{codegray},
    numbersep=5pt,
    tabsize=2,
}

\lstset{style=codestyle}
\newcommand{\code}[1]{\textcolor{codeorange}{\texttt{#1}}}
\newtcolorbox{quotebox}{
    colback=gray!5,
    colframe=gray!40,
    boxrule=0.5mm,
    arc=2mm,
    left=5mm,
    right=5mm
}

% ============================================
% HEADER / FOOTER
% ============================================
\setlength{\headheight}{14pt}
\pagestyle{fancy}
\fancyhf{}
\lhead{\leftmark}
\rfoot{\thepage}
\renewcommand{\sectionmark}[1]{\markboth{\MakeUppercase{#1}}{}}

% ============================================
% TITLE
% ============================================
\title{\textbf{Git and GitHub Essentials Summary}}
\author{Fhaheem Tadamarry}
\date{}

% ============================================
\begin{document}
\maketitle
\tableofcontents
\newpage

% ============================================
\refstepcounter{section}
\addcontentsline{toc}{section}{\protect\numberline{\thesection}INTRODUCTION TO GIT AND GITHUB}
\markboth{INTRODUCTION TO GIT AND GITHUB}{}

\subsection{Overview}
Git is a distributed version control system that tracks file changes locally, while GitHub is a cloud-based platform that hosts Git repositories. Together, they allow developers to version, back up, and collaborate on projects seamlessly.

\subsection{Key Concepts}
\begin{itemize}
    \item \textbf{Git}: Local version control — tracks and manages project changes.
    \item \textbf{GitHub}: Remote hosting service for Git repositories.
    \item \textbf{Repository (Repo)}: A folder containing all project files and Git history.
    \item \textbf{Commit}: A snapshot of your project’s state.
    \item \textbf{Push / Pull}: Send or retrieve commits between local and remote copies.
\end{itemize}

\begin{quotebox}
\textit{Git = your local brain, GitHub = your online memory.}
\end{quotebox}

% ============================================
\refstepcounter{section}
\addcontentsline{toc}{section}{\protect\numberline{\thesection}CREATING A LOCAL REPOSITORY}
\markboth{CREATING A LOCAL REPOSITORY}{}

\subsection{Initialize Git in a Folder}
To turn a regular folder into a Git repository:
\begin{lstlisting}[language=bash]
cd /path/to/project
git init
\end{lstlisting}

This creates a hidden \code{.git/} folder which stores the entire version history.

\subsection{Staging and Committing}
Add files to be tracked and save your first snapshot:
\begin{lstlisting}[language=bash]
git add .
git commit -m "initial commit"
\end{lstlisting}

Each commit acts as a restore point in your project’s history.

% ============================================
\refstepcounter{section}
\addcontentsline{toc}{section}{\protect\numberline{\thesection}CONNECTING TO GITHUB}
\markboth{CONNECTING TO GITHUB}{}

\subsection{Create and Link Remote Repository}
\begin{lstlisting}[language=bash]
git remote add origin https://github.com/fhaheem/EEL-6764.git
git branch -M main
git push -u origin main
\end{lstlisting}

This links your local project to the remote GitHub repository.

\subsection{Checking Connection}
\begin{lstlisting}[language=bash]
git remote -v
\end{lstlisting}

Displays the URLs used for pushing and pulling.

% ============================================
\refstepcounter{section}
\addcontentsline{toc}{section}{\protect\numberline{\thesection}DAILY WORKFLOW}
\markboth{DAILY WORKFLOW}{}

\subsection{Typical Daily Commands}
\begin{itemize}
    \item Pull the latest version from GitHub:
    \begin{lstlisting}[language=bash]
    git pull origin main
    \end{lstlisting}

    \item Stage and commit your local changes:
    \begin{lstlisting}[language=bash]
    git add .
    git commit -m "update: homework revisions"
    \end{lstlisting}

    \item Push changes back to GitHub:
    \begin{lstlisting}[language=bash]
    git push
    \end{lstlisting}
\end{itemize}

\subsection{Selective Add Options}
\begin{itemize}
    \item \code{git add .} → add all new + modified files.
    \item \code{git add -u} → add only modified/deleted files.
    \item \code{git add file.tex} → add a specific file.
\end{itemize}

% ============================================
\refstepcounter{section}
\addcontentsline{toc}{section}{\protect\numberline{\thesection}GITIGNORE AND CLEANUP}
\markboth{GITIGNORE AND CLEANUP}{}

\subsection{Purpose}
The \code{.gitignore} file tells Git which files to skip, keeping your repository clean.

\subsection{LaTeX Example}
\begin{lstlisting}[language=bash]
# macOS
.DS_Store

# LaTeX build files
*.aux
*.bbl
*.blg
*.fdb_latexmk
*.fls
*.log
*.out
*.synctex.gz
*.toc
*.pdf
\end{lstlisting}

\subsection{Removing Already-Tracked Junk}
\begin{lstlisting}[language=bash]
git rm -r --cached .
git add .
git commit -m "cleanup: remove ignored build files"
git push
\end{lstlisting}

% ============================================
\refstepcounter{section}
\addcontentsline{toc}{section}{\protect\numberline{\thesection}PULLING AND CLONING}
\markboth{PULLING AND CLONING}{}

\subsection{Cloning a Repo from GitHub}
To download (clone) a repo onto your computer:
\begin{lstlisting}[language=bash]
git clone https://github.com/fhaheem/EEL-6764.git
cd EEL-6764
\end{lstlisting}

\subsection{Pulling Updates}
To update your local copy with the latest online changes:
\begin{lstlisting}[language=bash]
git pull origin main
\end{lstlisting}

If you made local edits first:
\begin{lstlisting}[language=bash]
git add .
git commit -m "save local work"
git pull origin main
\end{lstlisting}

% ============================================
\refstepcounter{section}
\addcontentsline{toc}{section}{\protect\numberline{\thesection}STOPPING OR DISCONNECTING GIT}
\markboth{STOPPING OR DISCONNECTING GIT}{}

\subsection{Remove Git Tracking Locally}
If you want to keep your files but stop tracking:
\begin{lstlisting}[language=bash]
rm -rf .git
\end{lstlisting}

\subsection{Disconnect from GitHub but Keep Git}
\begin{lstlisting}[language=bash]
git remote remove origin
\end{lstlisting}

\subsection{Delete the Entire Folder}
\begin{lstlisting}[language=bash]
rm -rf foldername
\end{lstlisting}

% ============================================
\refstepcounter{section}
\addcontentsline{toc}{section}{\protect\numberline{\thesection}SUMMARY AND BEST PRACTICES}
\markboth{SUMMARY AND BEST PRACTICES}{}

\subsection{Daily Workflow Recap}
\begin{lstlisting}[language=bash]
# Start your day
git pull origin main

# After editing files
git add .
git commit -m "updated homework and README"

# Upload to GitHub
git push
\end{lstlisting}

\subsection{Key Tips}
\begin{itemize}
    \item Always pull before you start editing.
    \item Write clear commit messages (what + why).
    \item Use \code{.gitignore} to avoid committing junk.
    \item Never delete your GitHub repo unless intentional.
    \item Git only uploads differences — not the full project.
\end{itemize}

\begin{quotebox}
Once you master \texttt{git add}, \texttt{git commit}, \texttt{git push}, and \texttt{git pull},\\
you’ve learned 90\% of daily Git usage.
\end{quotebox}

% ============================================
\refstepcounter{section}
\addcontentsline{toc}{section}{\protect\numberline{\thesection}BRANCHING WORKFLOW}
\markboth{BRANCHING WORKFLOW}{}
% ============================================

\subsection{Why Branches?}
Branches let you create a safe, parallel line of development away from \code{main}. You can experiment, write HW solutions, or refactor without breaking the stable version. When finished, merge back into \code{main}.

\begin{quotebox}
Work on \code{main} for reading and releases; create short-lived feature branches for changes.
\end{quotebox}

\subsection{Create a Branch from \code{main}}
\begin{lstlisting}[language=bash]
# Make sure main is up to date
git checkout main
git pull origin main

# Create and switch to a new branch (e.g., HW5 work)
git switch -c hw5-development
# (older syntax) git checkout -b hw5-development
\end{lstlisting}

\subsection{Work, Commit, and Push Your Branch}
\begin{lstlisting}[language=bash]
# Edit files locally, then save your work
git add .
git commit -m "feat(hw5): initial draft of solutions"

# Publish the branch to GitHub (first push sets upstream)
git push -u origin hw5-development
\end{lstlisting}

\subsection{Keep Your Branch Fresh}
\begin{lstlisting}[language=bash]
# Bring latest main into your branch (rebases your commits on top)
git checkout hw5-development
git fetch origin
git rebase origin/main   # or: git merge origin/main
\end{lstlisting}

\subsection{Merge Back to \code{main} (CLI)}
\begin{lstlisting}[language=bash]
# Switch to main and ensure it's current
git checkout main
git pull origin main

# Merge the branch and push the result
git merge hw5-development
git push
\end{lstlisting}

\subsection{Merge Back to \code{main} (GitHub UI)}
Open a Pull Request (PR) from \code{hw5-development} into \code{main} on GitHub, review the diff, then click \textit{Merge}. This keeps a clean, auditable history.

\subsection{Clean Up (Optional)}
\begin{lstlisting}[language=bash]
# Delete the local branch
git branch -d hw5-development

# Delete the remote branch
git push origin --delete hw5-development
\end{lstlisting}

\subsection{Visual Model}
\begin{lstlisting}
main:     A --- B --- C -------- D --- E
                   \ 
branch:              X --- Y --- Z  (merge -> E)
\end{lstlisting}

\subsection{Quick Reference}
\begin{itemize}
  \item \code{git switch -c <name>} – create and switch to a new branch.
  \item \code{git push -u origin <name>} – publish branch to GitHub and set upstream.
  \item \code{git rebase origin/main} – replay your commits on top of latest \code{main}.
  \item \code{git merge <name>} – merge the named branch into the current branch.
  \item \code{git branch -d <name>} – delete local branch after merging.
\end{itemize}

\end{document}